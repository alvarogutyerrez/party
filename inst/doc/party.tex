
\documentclass[letter]{article}
\usepackage{amstext}
\usepackage{amsfonts}
\usepackage{hyperref}
\usepackage[round]{natbib}
\usepackage{hyperref}


%%\VignetteIndexEntry{party: A Laboratory for Recursive Part(y)itioning}
%%\VignetteDepends{coin,ipred}

\usepackage{verbatim}
\usepackage{graphicx}
\usepackage{amsfonts}
\usepackage{amstext}
\usepackage{amsmath}
%%\usepackage{amsthm}
%%\usepackage[round]{natbib}
%%\usepackage{bibentry}
%%\usepackage{hyperref}
\usepackage{thumbpdf}
\usepackage{rotating}
%%\usepackage{floatflt}

\newcommand{\R}{\mathbb{R} }
\newcommand{\Prob}{\mathbb{P} }
\newcommand{\N}{\mathbb{N} }
\newcommand{\C}{\mathbb{C} }
\newcommand{\V}{\mathbb{V}} %% cal{\mbox{\textnormal{Var}}} }
\newcommand{\E}{\mathbb{E}} %%mathcal{\mbox{\textnormal{E}}} }
\newcommand{\Var}{\mathbb{V}} %%mathcal{\mbox{\textnormal{Var}}} }
\newcommand{\argmin}{\operatorname{argmin}\displaylimits}
\newcommand{\argmax}{\operatorname{argmax}\displaylimits}
\newcommand{\LS}{\mathcal{L}_n}
\newcommand{\TS}{\mathcal{T}_n}
\newcommand{\LSc}{\mathcal{L}_{\text{comb},n}}
\newcommand{\LSbc}{\mathcal{L}^*_{\text{comb},n}}
\newcommand{\F}{\mathcal{F}}
\newcommand{\A}{\mathcal{A}}
\newcommand{\yn}{y_{\text{new}}}
\newcommand{\z}{\mathbf{z}}
\newcommand{\X}{\mathbf{X}}
\newcommand{\Y}{\mathbf{Y}}
\newcommand{\sX}{\mathcal{X}}
\newcommand{\sY}{\mathcal{Y}}
\newcommand{\T}{\mathbf{T}}
\newcommand{\x}{\mathbf{x}}
\renewcommand{\a}{\mathbf{a}}
\newcommand{\xn}{\mathbf{x}_{\text{new}}}
\newcommand{\y}{\mathbf{y}}
\newcommand{\w}{\mathbf{w}}
\newcommand{\ws}{\mathbf{w}_\cdot}
\renewcommand{\t}{\mathbf{t}}
\newcommand{\M}{\mathbf{M}}
\renewcommand{\vec}{\text{vec}}
\newcommand{\B}{\mathbf{B}}
\newcommand{\K}{\mathbf{K}}
\newcommand{\W}{\mathbf{W}}
\newcommand{\D}{\mathbf{D}}
\newcommand{\I}{\mathbf{I}}
\newcommand{\bS}{\mathbf{S}}
\newcommand{\cellx}{\pi_n[\x]}
\newcommand{\partn}{\pi_n(\mathcal{L}_n)}
\newcommand{\err}{\text{Err}}
\newcommand{\ea}{\widehat{\text{Err}}^{(a)}}
\newcommand{\ecv}{\widehat{\text{Err}}^{(cv1)}}
\newcommand{\ecvten}{\widehat{\text{Err}}^{(cv10)}}
\newcommand{\eone}{\widehat{\text{Err}}^{(1)}}
\newcommand{\eplus}{\widehat{\text{Err}}^{(.632+)}}
\newcommand{\eoob}{\widehat{\text{Err}}^{(oob)}}

\hyphenation{Qua-dra-tic}


\hypersetup{%
  pdftitle = {party: A Laboratory for Recursive Part(y)itioning},
  pdfsubject = {Manuscript},
  pdfauthor = {Torsten Hothorn, Kurt Hornik, and Achim Zeileis},
%% change colorlinks to false for pretty printing
  colorlinks = {true},
  linkcolor = {blue},
  citecolor = {blue},
  urlcolor = {red},
  hyperindex = {true},
  linktocpage = {true},
}

\newcommand{\bft}{\mathbf{t}}

\usepackage{/usr/local/lib/R/share/texmf/Sweave}
\begin{document}

\title{\texttt{party}: A Laboratory for Recursive Part(y)itioning}
\author{Torsten Hothorn$^1$, Kurt Hornik$^2$ and Achim Zeileis$^2$}
\date{}
\maketitle

\noindent$^1$Institut f\"ur Medizininformatik, Biometrie und Epidemiologie\\
     Friedrich-Alexander-Universit\"at Erlangen-N\"urnberg\\
     Waldstra{\ss}e 6, D-91054 Erlangen, Germany \\
     \texttt{Torsten.Hothorn@R-project.org}
\newline

\noindent$^2$Institut f\"ur Statistik und Mathematik,
             Wirtschaftsuniversit\"at Wien \\
       Augasse 2-6, A-1090 Wien, Austria \\
       \texttt{Kurt.Hornik@R-project.org} \\
       \texttt{Achim.Zeileis@R-project.org} 
\newline




\setkeys{Gin}{width=\textwidth}


\section{Introduction}

The majority of recursive partitioning algorithms 
are special cases of a simple two-stage algorithm: 
First partition the observations by univariate splits in a recursive way and
second fit a constant model in each cell of the resulting partition.
The most popular implementations of such algorithms are `CART' 
\citep{classifica:1984} and `C4.5' \citep{Quinlan1993}. Not unlike AID,   
both perform an exhaustive search over all possible splits maximizing an
information measure of node impurity selecting the 
covariate showing the best split.
This approach has two fundamental problems: overfitting and a selection 
bias towards covariates with many possible splits. 
With respect to the 
overfitting problem \cite{Mingers1987} notes that the algorithm
\begin{quote}
[\ldots] has no concept of statistical significance, and so cannot
distinguish between a significant and an insignificant improvement in the
information measure.
\end{quote}
With the \texttt{party} package 
we enter at the point where \cite{WhiteLiu1994} demand for
\begin{quote}
[\ldots] a \textit{statistical} approach [to recursive partitioning] which
takes
into account the \textit{distributional} properties of the measures.
\end{quote} 
We present a unified framework embedding recursive binary partitioning into
the well defined theory of permutation tests developed by
\cite{StrasserWeber1999}.
The conditional distribution of statistics measuring the association between
responses and covariates is the basis for an unbiased selection among
covariates measured at different scales. 
Moreover, multiple test procedures are applied to determine whether no
significant
association between any of the covariates and the response can be stated and 
the recursion needs to stop.

\section{Recursive Binary Partitioning \label{algo}}

We focus on regression models describing the conditional distribution of a
response variable $\Y$ given the status of $m$ covariates by
means of tree-structured recursive partitioning. The response $\Y$ from some
sample space $\sY$ may be multivariate as well. 
The $m$-dimensional covariate vector $\X = (X_1, \dots, X_m)$ is taken 
from a sample space $\sX = \sX_1 \times \cdots \times \sX_m$.
Both response variable and covariates may be measured
at arbitrary scales.
We assume that the conditional distribution $D(\Y | \X)$ of the response 
$\Y$ given the covariates $\X$ depends on a function $f$ of the covariates
\begin{eqnarray*} 
D(\Y | \X) = D(\Y | X_1, \dots, X_m) = D(\Y | f( X_1, \dots,
X_m)),
\end{eqnarray*}
where we restrict ourselves to partition based regression relationships,
i.e., $r$ disjoint cells $B_1, \dots, B_r$ partitioning the covariate space $\sX
= \bigcup_{k = 1}^r B_k$.
A model of the regression relationship is to be fitted based on a learning 
sample $\LS$, i.e., a random sample of $n$ independent and
identically distributed observations, possibly with some covariates $X_{ji}$
missing,
\begin{eqnarray*}
\LS & = & \{ (\Y_i, X_{1i}, \dots, X_{mi}); i = 1, \dots, n \}.
\end{eqnarray*}
For the sake of simplicity, we use a learning sample
\begin{Schunk}
\begin{Sinput}
> ls <- data.frame(y = gl(3, 50, labels = c("A", "B", 
+     "C")), x1 = rnorm(150) + rep(c(1, 0, 0), c(50, 
+     50, 50)), x2 = runif(150))
\end{Sinput}
\end{Schunk}
in the following illustrations.
A generic algorithm for recursive binary partitioning for a given learning
sample $\LS$ can be formulated using non-negative integer valued case weights $\w
= (w_1, \dots, w_n)$. Each node of a tree is represented by a vector of
case weights having non-zero elements when the corresponding observations
are elements of the node and are zero otherwise. The following algorithm
implements recursive binary partitioning:
\begin{enumerate}
\item For case weights $\w$ test the global null hypothesis of independence between
      any of the $m$ covariates and the response. Stop if this 
      hypothesis cannot be rejected. 
      Otherwise select the covariate $X_{j^*}$ with strongest 
      association to $\Y$.
\item Choose a set $A^* \subset \sX_{j^*}$ in order to split $\sX_{j^*}$ into
      two disjoint sets $A^*$ and $\sX_{j^*} \setminus A^*$. 
      The case weights $\w_\text{left}$ and $\w_\text{right}$ determine the
      two subgroups with $w_{\text{left},i} = w_i I(X_{j^*i} \in A^*)$ and 
      $w_{\text{right},i} = w_i I(X_{j^*i} \not\in A^*)$ for all $i = 1,
      \dots, n$ ($I(\cdot)$ denotes the indicator function).
\item Recursively repeat steps 1 and 2 with modified case weights 
      $\w_\text{left}$ and $\w_\text{right}$, respectively.
\end{enumerate}
The separation of variable
selection and splitting procedure into steps 1 and 2 of the algorithm
is the key for the construction of interpretable tree
structures not suffering a systematic tendency towards covariates with many
possible splits or many missing values. In addition, a statistically
motivated and intuitive stopping criterion can be implemented: We stop 
when the global null hypothesis of independence between the
response and any of the $m$ covariates cannot be rejected at a pre-specified
nominal level~$\alpha$. The algorithm induces a partition $\{B_1, \dots, B_r\}$ of
the covariate space $\sX$, where each cell $B \in \{B_1, \dots, B_r\}$ 
is associated with a vector of case weights. 

In package \texttt{party}, the dependency structure and the variables may be
specified in a traditional formula based way
\begin{Schunk}
\begin{Sinput}
> library("party")
> ctree(y ~ x1 + x2, data = ls)
\end{Sinput}
\end{Schunk}
Case counts $\w$ may be specified using the \texttt{weights} argument.

\section{Recursive Partitioning by Conditional Inference \label{framework}}

In the main part of this section we focus on step 1 of the generic algorithm.
Unified tests for independence are constructed by means of the conditional
distribution of linear statistics in the permutation test framework
developed by \cite{StrasserWeber1999}. The determination of the best binary split
in one selected covariate and the handling of missing values
is performed based on standardized linear statistics within the same
framework as well. 

\paragraph{Variable Selection and Stopping Criteria.}
At step 1 of the generic algorithm given in Section~\ref{algo} we face an 
independence problem. We need to decide whether there is any information
about the response variable covered by any of the $m$  covariates. In each node
identified by case weights $\w$, the
global hypothesis of independence is formulated in terms of the $m$ partial hypotheses
$H_0^j: D(\Y | X_j) = D(\Y)$ with global null hypothesis $H_0 = \bigcap_{j = 1}^m
H_0^j$.
When we are not able to reject $H_0$ at a pre-specified 
level $\alpha$, we stop the recursion.
If the global hypothesis can be rejected, we measure the association
between $\Y$ and each of the covariates $X_j, j = 1, \dots, m$, by
test statistics or $P$-values indicating the deviation from the partial
hypotheses $H_0^j$.  

For notational convenience and without loss of generality we assume that the
case weights $w_i$ are either zero or one. The symmetric group of all
permutations of  the elements of $(1, \dots, n)$ with corresponding case
weights $w_i = 1$ is denoted by $S(\LS, \w)$. A more general notation is
given in the Appendix. We measure the association between $\Y$ and $X_j, j = 1, \dots, m$, 
by linear statistics of the form
\begin{eqnarray} \label{linstat}
\T_j(\LS, \w) = \vec \left( \sum_{i=1}^n w_i g_j(X_{ji})
h(\Y_i, (\Y_1, \dots, \Y_n))^\top \right) \in \R^{p_jq}
\end{eqnarray}
where $g_j: \sX_j \rightarrow \R^{p_j}$ is a non-random transformation of
the covariate $X_j$. The transformation may be specified using the
\texttt{xtrafo} argument. If, for example, a ranking \textit{both}
\texttt{x1} and \texttt{x2} is required,
\begin{Schunk}
\begin{Sinput}
> ctree(y ~ x1 + x2, data = ls, xtrafo = function(data) trafo(data, 
+     numeric_trafo = rank))
\end{Sinput}
\end{Schunk}
can be used. The \emph{influence function} 
$h: \sY \times \sY^n \rightarrow
\R^q$ depends on the responses $(\Y_1, \dots, \Y_n)$ in a permutation
symmetric way. 
%%, i.e., $h(\Y_i, (\Y_1, \dots, \Y_n)) = h(\Y_i, (\Y_{\sigma(1)}, \dots,
%%\Y_{\sigma(n)}))$ for all permutations $\sigma \in S(\LS, \w)$. 
Section~\ref{examples} explains how to choose $g_j$ and $h$ in different 
practical settings. A $p_j \times q$ matrix is converted into a 
$p_jq$ column vector by column-wise combination using the `vec' operator. 
The influence function can be specified using the \texttt{ytrafo} argument.

The distribution of $\T_j(\LS, \w)$ under $H_0^j$ depends on the joint
distribution of $\Y$ and $X_j$, which is unknown under almost all practical
circumstances. At least under the null hypothesis one can dispose of this
dependency by fixing the covariates and conditioning on all possible 
permutations of the responses. This principle leads to test procedures known
as \textit{permutation tests}. 
The conditional expectation $\mu_j \in \R^{p_jq}$ and covariance $\Sigma_j
\in \R^{p_jq \times p_jq}$ of $\T_j(\LS, \w)$ under $H_0$ given
all permutations $\sigma \in S(\LS, \w)$ of the responses are derived by
\cite{StrasserWeber1999}: 
\begin{eqnarray}
\mu_j & = & \E(\T_j(\LS, \w) | S(\LS, \w)) = 
\vec \left( \left( \sum_{i = 1}^n w_i g_j(X_{ji}) \right) \E(h | S(\LS,
\w))^\top
\right), \nonumber \\
\Sigma_j & = & \V(\T_j(\LS, \w) | S(\LS, \w)) \nonumber \\
& = & 
    \frac{\ws}{\ws - 1}  \V(h | S(\LS, \w)) \otimes
        \left(\sum_i w_i  g_j(X_{ji}) \otimes w_i  g_j(X_{ji})^\top \right) \label{expectcovar}
\\
& - & \frac{1}{\ws - 1}  \V(h | S(\LS, \w))  \otimes \left(
        \sum_i w_i g_j(X_{ji}) \right)
\otimes \left( \sum_i w_i g_j(X_{ji})\right)^\top 
\nonumber
\end{eqnarray}
where $\ws = \sum_{i = 1}^n w_i$ denotes the sum of the case weights,
$\otimes$ is the Kronecker product and the conditional expectation of the
influence function is 
\begin{eqnarray*}
\E(h | S(\LS, \w)) = \ws^{-1} \sum_i w_i h(\Y_i, (\Y_1, \dots, \Y_n)) \in
\R^q
\end{eqnarray*} 
with corresponding $q \times q$ covariance matrix
\begin{eqnarray*}
\V(h | S(\LS, \w)) = \ws^{-1} \sum_i w_i \left(h(\Y_i, (\Y_1, \dots, \Y_n)) - \E(h | S(\LS, \w))
\right) \\
\left(h(\Y_i, (\Y_1, \dots, \Y_n)) - \E(h | S(\LS, \w))\right)^\top.
\end{eqnarray*}
Having the conditional expectation and covariance at hand we are able to
standardize a linear statistic $\T \in \R^{pq}$ of the form
(\ref{linstat}) for some $p \in \{p_1, \dots, p_m\}$. 
Univariate test statistics~$c$ mapping an observed multivariate 
linear statistic $\bft \in
\R^{pq}$ into the real line can be of arbitrary form.  An obvious choice is
the maximum of the absolute values of the standardized linear statistic
\begin{eqnarray*}
c_\text{max}(\bft, \mu, \Sigma)  = \max_{k = 1, \dots, pq} \left| \frac{(\bft -
\mu)_k}{\sqrt{(\Sigma)_{kk}}} \right|
\end{eqnarray*}
utilizing the conditional expectation $\mu$ and covariance matrix
$\Sigma$. The application of a quadratic form $c_\text{quad}(\bft, \mu, \Sigma)  = 
(\bft - \mu) \Sigma^+ (\bft - \mu)^\top$ is one alternative, although
computationally more expensive because the Moore-Penrose 
inverse $\Sigma^+$ of $\Sigma$ is involved.

The type of test statistic to be used can be specified by means of the
\texttt{ctree\_control} function, for example
\begin{Schunk}
\begin{Sinput}
R> ctree(y ~ x1 + x2, data = ls, 
      control = ctree_control(teststattype = "maxabs"))
\end{Sinput}
\end{Schunk}
uses $c_\text{max}$ and 
\begin{Schunk}
\begin{Sinput}
R> ctree(y ~ x1 + x2, data = ls, 
      control = ctree_control(teststattype = "quadform"))
\end{Sinput}
\end{Schunk}
takes $c_\text{quad}$ (the default).

It is important to note that the test statistics $c(\bft_j, \mu_j, \Sigma_j),
j = 1, \dots, m$, cannot be directly compared in an unbiased  
way unless all of the covariates are measured at the same scale, i.e.,
$p_1 = p_j, j = 2, \dots, m$. 
In order to allow for an unbiased variable selection we need to switch to
the $P$-value scale because $P$-values for the conditional distribution of test
statistics $c(\T_j(\LS, \w), \mu_j, \Sigma_j)$ 
can be directly compared among covariates
measured at different scales. In step 1 of the generic algorithm we 
select the covariate with minimum $P$-value, i.e., the covariate $X_{j^*}$
with $j^*  =  \argmin_{j = 1, \dots, m} P_j$, where
\begin{displaymath}
P_j  =  \Prob_{H_0^j}(c(\T_j(\LS, \w), \mu_j,
            \Sigma_j) \ge c(\bft_j, \mu_j, \Sigma_j) | S(\LS, \w))
\end{displaymath}
denotes the $P$-value of the conditional test for $H_0^j$. 
So far, we have only addressed testing each partial hypothesis $H_0^j$, which
is sufficient for an unbiased variable selection. A global test 
for $H_0$ required in step 1 can be constructed via an aggregation of the 
transformations $g_j, j = 1, \dots, m$,
i.e., using a linear statistic of the form
\begin{eqnarray*}
\T(\LS, \w) = \vec \left( \sum_{i=1}^n w_i
\left(g_1(X_{1i})^\top, \dots,
g_m(X_{mi})^\top\right)^\top h(\Y_i, (\Y_1, \dots, \Y_n))^\top \right).
%%\in \R^{\sum_j p_jq}
\end{eqnarray*}
However, this approach is less attractive for learning samples with
missing values. Universally applicable approaches are multiple test 
procedures based on $P_1, \dots, P_m$. 
Simple Bonferroni-adjusted $P$-values $m
P_j$, available via 
\begin{Schunk}
\begin{Sinput}
> ctree_control(testtype = "Bonferroni")
\end{Sinput}
\end{Schunk}
or a min-$P$-value resampling approach 
\begin{Schunk}
\begin{Sinput}
> ctree_control(testtype = "MonteCarlo")
\end{Sinput}
\end{Schunk}
are just examples and we refer
to the multiple testing literature \citep[e.g.,][]{WestfallYoung1993}
for more advanced methods. We reject $H_0$ when the minimum 
of the adjusted $P$-values is less than a pre-specified nominal level $\alpha$
and otherwise stop the algorithm. In this sense, $\alpha$
may be seen as a unique parameter determining the size of the resulting trees.

\paragraph{Splitting Criteria.}
Once we have selected a covariate in step 1 of the algorithm, the split itself can be
established by any split criterion, including those established by
\cite{classifica:1984} or \cite{Shih1999}. Instead of simple binary splits, 
multiway splits can be implemented as well, for example utilizing
the work of \cite{OBrien2004}. However, most splitting criteria are not
applicable to response variables measured at arbitrary scales and we
therefore utilize the permutation test framework described above  
to find the optimal binary split in one selected 
covariate $X_{j^*}$ in step~2 of the generic algorithm. The goodness of a split is
evaluated by two-sample linear statistics which are special cases of the linear 
statistic (\ref{linstat}). 
For all possible subsets $A$ of the sample space $\sX_{j^*}$ the linear
statistic
\begin{eqnarray*}
\T^A_{j^*}(\LS, \w) = \vec \left( \sum_{i=1}^n w_i I(X_{j^*i} \in A) 
                                  h(\Y_i, (\Y_1, \dots, \Y_n))^\top \right) \in \R^{q}
\end{eqnarray*}
induces a two-sample statistic measuring the discrepancy between the samples 
$\{ \Y_i | w_i > 0 \text{ and } X_{ji} \in A; i = 1, \dots, n\}$ 
and $\{ \Y_i | w_i > 0 \text{ and } X_{ji} \not\in A; i = 1, \dots, n\}$. 
The conditional expectation $\mu_{j^*}^A$ and covariance
$\Sigma_{j^*}^A$
can be computed by (\ref{expectcovar}).
The split $A^*$ with a test statistic maximized over all possible
subsets $A$ is established:
\begin{eqnarray} \label{split}
A^* = \argmax_A c(\bft_{j^*}^A, \mu_{j^*}^A, \Sigma_{j^*}^A).
\end{eqnarray}
The statistics $c(\bft_{j^*}^A, \mu_{j^*}^A, \Sigma_{j^*}^A)$ are available
for each node with
\begin{Schunk}
\begin{Sinput}
> ctree_control(savesplitstats = TRUE)
\end{Sinput}
\end{Schunk}
and can be used to depict a scatter plot of the covariate
$\sX_{j^*}$ against the statistics.

Note that we do not need to compute the distribution of 
$c(\bft_{j^*}^A, \mu_{j^*}^A, \Sigma_{j^*}^A)$ in step~2. 
In order to anticipate pathological splits one can restrict the number of
possible subsets that are evaluated, for example by introducing restrictions
on the sample size or the sum of the case weights in each of the two groups
of observations induced by a possible split. For example,
\begin{Schunk}
\begin{Sinput}
> ctree_control(minsplit = 20)
\end{Sinput}
\end{Schunk}
requires the sum of the weights in both the left and right daughter 
node to exceed the value of $20$.

\paragraph{Missing Values and Surrogate Splits.}

If an observation $X_{ji}$ in covariate $X_j$ is missing, we set the
corresponding case weight $w_i$ to zero for the computation of $\T_j(\LS, \w)$
and, if we would like to split in $X_j$, in $\T_{j}^A(\LS, \w)$ as well.
Once a split $A^*$ in $X_j$ has been implemented, surrogate splits can be 
established by
searching for a split leading to roughly the same division of the
observations as the original split. One simply replaces the original
response variable by a binary variable $I(X_{ji} \in A^*)$ coding the
split and proceeds as described in the previous part. The number of
surrogate splits can be controlled using
\begin{Schunk}
\begin{Sinput}
> ctree_control(maxsurrogate = 3)
\end{Sinput}
\end{Schunk}

\paragraph{Inspecting a Tree.}

Once we have fitted a conditional tree via
\begin{Schunk}
\begin{Sinput}
> ct <- ctree(y ~ x1 + x2, data = ls)
\end{Sinput}
\end{Schunk}
we can inspect the results via a \texttt{print} method
\begin{Schunk}
\begin{Sinput}
> ct
\end{Sinput}
\begin{Soutput}
	 Conditional tree with 2 terminal nodes

Response:  y 
Inputs:  x1, x2 
Number of observations:  150 

1) x1 <= 0.8255248; criterion = 1, statistic = 22.991
  2)*  weights = 96 
1) x1 > 0.8255248
  3)*  weights = 54 
\end{Soutput}
\end{Schunk}
or by looking at a graphical representation as in Figure \ref{party-plot1}.

\begin{figure}[t]
\begin{center}
\begin{Schunk}
\begin{Sinput}
> plot(ct)
\end{Sinput}
\end{Schunk}
\includegraphics{party-party-plot}
\caption{A graphical representation of a classification tree.
\label{party-plot1}}
\end{center}
\end{figure}

Each node can be extracted by its node number, i.e., the root node is
\begin{Schunk}
\begin{Sinput}
> nodes(ct, 1)
\end{Sinput}
\begin{Soutput}
[[1]]
1) x1 <= 0.8255248; criterion = 1, statistic = 22.991
  2)*  weights = 96 
1) x1 > 0.8255248
  3)*  weights = 54 
\end{Soutput}
\end{Schunk}
This object is a conventional list with elements
\begin{Schunk}
\begin{Sinput}
> names(nodes(ct, 1)[[1]])
\end{Sinput}
\begin{Soutput}
[1] "nodeID"     "weights"    "criterion"  "terminal"  
[5] "psplit"     "ssplits"    "prediction" "left"      
[9] "right"     
\end{Soutput}
\end{Schunk}
and we refer to the manual pages for a description of those elements.
The \texttt{Predict} function aims at computing predictions in the space of
the response variable, in our case a factor
\begin{Schunk}
\begin{Sinput}
> Predict(ct, newdata = ls)
\end{Sinput}
\begin{Soutput}
  [1] A A A A C A C A C C A A C A A A A C A C A A A C A A A C C A
 [31] A C A A C A A C C C A A C C C C A A A A A A C C C C A C C A
 [61] C C C C C C A A A A A C C A C A C C C C C C C C C C C C A C
 [91] A C A C C C C C C C C A C C C A C C A C C C C C C C A C C C
[121] C C C C C C C C C C C C C C C C C A C C C C A C C A C A C A
Levels: A B C
\end{Soutput}
\end{Schunk}
When we are interested in properties of the conditional distribution of the
response given the covariates, we use
\begin{Schunk}
\begin{Sinput}
> treeresponse(ct, newdata = ls[c(1, 51, 101), ])
\end{Sinput}
\begin{Soutput}
[[1]]
[1] 0.5740741 0.2592593 0.1666667

[[2]]
[1] 0.5740741 0.2592593 0.1666667

[[3]]
[1] 0.1979167 0.3750000 0.4270833
\end{Soutput}
\end{Schunk}
which, in our case, is a list with conditional class probabilities.
We can determine the node numbers of nodes some new observations are falling
into by
\begin{Schunk}
\begin{Sinput}
> where(ct, newdata = ls[c(1, 51, 101), ])
\end{Sinput}
\begin{Soutput}
[1] 3 3 2
\end{Soutput}
\end{Schunk}

\section{Examples \label{examples}}

\paragraph{Univariate Continuous or Discrete Regression.}
For a univariate numeric response $\Y \in \R$, the most natural influence function
is the identity $h(\Y_i, (\Y_1, \dots, \Y_n)) = \Y_i$. 
In case some observations with extremely large or small values have been
observed, a
ranking of the observations may be appropriate:
$h(\Y_i, (\Y_1, \dots, \Y_n)) = \sum_{k=1}^n w_k I(\Y_k \le \Y_i)$ for $i = 1, \dots,
n$.
Numeric covariates can be handled by the identity transformation 
$g_{ji}(x) = x$ (ranks are possible, too). Nominal covariates at levels $1,
\dots, K$ are
represented by $g_{ji}(k) = e_K(k)$, the unit vector of length $K$ with
$k$th element being equal to one. Due to this flexibility, special test 
procedures like the Spearman test, the 
Wilcoxon-Mann-Whitney test or the Kruskal-Wallis test and permutation tests
based on ANOVA statistics or correlation coefficients
are covered by this framework. Splits obtained from (\ref{split}) maximize the
absolute value of the standardized difference between two means of the
values of the influence functions. 
For prediction, one is usually interested in an estimate of 
the expectation of
the response $\E(\Y | \X = \x)$ in each cell, an estimate can be 
obtained by 
\begin{eqnarray*}
\hat{\E}(\Y | \X = \x) = \left(\sum_{i=1}^n w_i(\x)\right)^{-1} \sum_{i=1}^n
w_i(\x) \Y_i.
\end{eqnarray*}

\paragraph{Censored Regression.}
The influence function $h$ may be chosen as 
Logrank or Savage scores taking censoring into
account and one can proceed as for univariate continuous regression. This is
essentially the approach first published by \cite{regression:1988}. 
An alternative is the weighting scheme suggested by
\cite{MolinaroDudiotvdLaan2003}. A weighted Kaplan-Meier curve for the case
weights $\w(\x)$ can serve as prediction. 

\paragraph{$J$-Class Classification.}
The nominal response variable at levels $1, \dots, J$ 
is handled by influence
functions $h(\Y_i, (\Y_1, \dots, \Y_n)) = e_J(\Y_i)$. Note that for a
nominal covariate $X_j$ at levels $1, \dots, K$ with  
$g_{ji}(k) = e_K(k)$ the
corresponding linear statistic $\T_j$ is a vectorized contingency table.
The conditional class probabilities can be estimated via 
\begin{eqnarray*}
\hat{\Prob}(\Y = y | \X = \x) = \left(\sum_{i=1}^n w_i(\x)\right)^{-1}
\sum_{i=1}^n
w_i(\x) I(\Y_i = y), \quad y = 1, \dots, J.
\end{eqnarray*}

\paragraph{Ordinal Regression.}
Ordinal response variables measured at $J$ levels, and ordinal covariates
measured at $K$ levels, are associated with score vectors $\xi \in
\R^J$ and $\gamma \in \R^K$, respectively. Those scores reflect the
`distances' between the levels: If the variable is derived from an
underlying continuous variable, the scores can be chosen as the midpoints
of the intervals defining the levels. The linear statistic is now a linear
combination of the linear statistic $\T_j$ of the form
\begin{eqnarray*}
\M \T_j(\LS, \w) & = & \vec \left( \sum_{i=1}^n w_i \gamma^\top g_j(X_{ji})
            \left(\xi^\top h(\Y_i, (\Y_1, \dots, \Y_n)\right)^\top \right)
\end{eqnarray*}
with $g_j(x) = e_K(x)$ and $h(\Y_i, (\Y_1, \dots, \Y_n)) = e_J(\Y_i)$.
If both response and covariate are ordinal, the matrix of coefficients
is given by the Kronecker product of both score vectors $\M = \xi \otimes \gamma \in
\R^{1, KJ}$. In case the response is ordinal only, the matrix of 
coefficients $\M$ is a block matrix 
\begin{eqnarray*}
\M = 
\left( 
    \begin{array}{ccc}
        \xi_1 &          & 0     \\
              &  \ddots  &       \\
          0   &          & \xi_1 
    \end{array} \right| 
%%\left.
%%    \begin{array}{ccc}
%%       \xi_2 &          & 0     \\
%%              &  \ddots  &       \\
%%          0   &          & \xi_2 
%%    \end{array} \right| 
%%\left.
    \begin{array}{c}
         \\
      \hdots \\
          \\
    \end{array} %%\right.
\left|
    \begin{array}{ccc}
        \xi_q &          & 0     \\
              &  \ddots  &       \\
          0   &          & \xi_q 
    \end{array} 
\right) 
%%\in \R^{K, KJ}
%%\end{eqnarray*}
\text{ or } %%and if one covariate is ordered 
%%\begin{eqnarray*}
%%\M = \left( 
%%    \begin{array}{cccc}
%%        \gamma & 0      &        & 0 \\
%%        0      & \gamma &        & \vdots \\
%%       0      & 0      & \hdots & 0 \\
%%        \vdots & \vdots &        & 0 \\
%%        0      & 0      &        & \gamma
%%    \end{array} 
%%    \right) 
\M = \text{diag}(\gamma)
%%\in \R^{J, KJ}
\end{eqnarray*}
when one covariate is ordered but the response is not. For both $\Y$ and $X_j$
being ordinal, the corresponding test is known as linear-by-linear association
test \citep{Agresti2002}. Scores can be supplied to \texttt{ctree} using the
\texttt{scores} argument, see Section \ref{illustrations} for an example.

\paragraph{Multivariate Regression.}
For multivariate responses, the influence function is a combination of
influence functions appropriate for any of the univariate response variables
discussed in the previous paragraphs, e.g., indicators for multiple binary
responses \citep{Zhang1998,NohSongPark2004}, Logrank or Savage scores
for multiple failure times 
%%\citep[for example tooth loss times, ][]{SuFan2004}
and the original observations or a rank transformation for multivariate regression 
\citep{Death2002}.

\section{Illustrations and Applications \label{illustrations}}

In this section, we present regression problems which illustrate the
potential fields of application of the methodology.  
Conditional inference trees based on $c_\text{quad}$-type test statistics 
using the identity influence function for numeric responses 
and asymptotic $\chi^2$ distribution are applied.
For the stopping criterion a simple
Bonferroni correction is used and we follow the usual convention by choosing
the nominal level of the conditional independence tests as $\alpha = 0.05$.

\paragraph{Tree Pipit Abundance.}

\begin{Schunk}
\begin{Sinput}
> data(treepipit, package = "coin")
> tptree <- ctree(counts ~ ., data = treepipit)
\end{Sinput}
\end{Schunk}

\begin{figure}[t]
\begin{center}
\begin{Schunk}
\begin{Sinput}
> plot(tptree, terminal_panel = node_hist(tptree, breaks = 0:6 - 
+     0.5, ymax = 65, horizontal = FALSE, freq = TRUE))
\end{Sinput}
\end{Schunk}
\includegraphics{party-treepipit-plot}
\caption{Conditional regression tree for the tree pipit data.}
\end{center}
\end{figure}


The impact of certain environmental factors on the population density of the 
tree pipit \textit{Anthus trivialis} 
%%in Frankonian oak forests 
is investigated by \cite{MuellerHothorn2004}.
The occurrence of tree pipits was recorded several times at
$n = 86$ stands which were established on a long environmental gradient. 
Among nine environmental factors, 
the covariate showing the largest association to the number of tree pipits
is the canopy overstorey $(P = $0.002$)$. 
Two groups of stands can be distinguished: Sunny stands with less than $40\%$
canopy overstorey $(n = 24)$ show a
significantly higher density of tree pipits compared to darker stands with more than
$40\%$ canopy overstorey $(n = 62)$.
This result is important for management decisions
in forestry enterprises: Cutting the overstorey with release of old
oaks creates a perfect habitat for this indicator species
of near natural forest environments. 

\paragraph{Glaucoma \& Laser Scanning Images.}

\begin{Schunk}
\begin{Sinput}
> data(GlaucomaM, package = "ipred")
> gtree <- ctree(Class ~ ., data = GlaucomaM)
\end{Sinput}
\end{Schunk}


Laser scanning images taken from the eye background are expected to serve as 
the basis of an automated system for glaucoma diagnosis. 
Although prediction is more important in this application
\citep{new-glauco:2003}, a simple
visualization of the regression relationship is useful for comparing the
structures inherent in the learning sample with subject matter knowledge. 
For $98$ patients and $98$ controls, matched by age
and gender, $62$ covariates describing the eye morphology are available. The
data is part of the 
\texttt{ipred} package
(\cite{Rnews:Peters+Hothorn+Lausen:2002}, \url{http://CRAN.R-project.org}).
The first split in Figure~\ref{glaucoma} separates eyes with a volume 
above reference less than
$0.059 \text{ mm}^3$ in the inferior part of the
optic nerve head (\texttt{vari}). 
Observations with larger volume are mostly controls, a finding which
corresponds to subject matter knowledge: The volume above reference measures the
thickness of the nerve layer, expected to decrease with a glaucomatous
damage of the optic nerve. Further separation is achieved by the volume
above surface global (\texttt{vasg}) and the volume above reference in the
temporal part of the optic nerve head (\texttt{vart}).

\begin{figure}[t]
\begin{center}
\begin{Schunk}
\begin{Sinput}
> plot(gtree)
\end{Sinput}
\end{Schunk}
\includegraphics{party-glaucoma-plot}
\caption{Conditional inference tree for the glaucoma data. For each inner node, the
Bonferroni-adjusted $P$-values are given, the fraction of glaucomatous eyes
is displayed for each terminal node. \label{glaucoma}}
\end{center}
\end{figure}

The plot in Figure~\ref{glaucoma} shows the distribution of the classes in 
the terminal nodes. This distribution can be shown for the inner nodes as
well, namely by specifying the appropriate panel generating function
(\texttt{node\_barplot} in our case), see Figure~\ref{glaucoma-inner}.

\begin{figure}[ht!]
\begin{center}
\begin{Schunk}
\begin{Sinput}
> plot(gtree, inner_panel = node_barplot, edge_panel = function(ctreeobj, 
+     ...) {
+     function(...) invisible()
+ }, tnex = 1)
\end{Sinput}
\end{Schunk}
\includegraphics{party-glaucoma-plot-inner}
\caption{Conditional inference tree for the glaucoma data with the 
         fraction of glaucomatous eyes displayed for both inner and terminal
         nodes. \label{glaucoma-inner}}
\end{center}
\end{figure}


As mentioned in Section \ref{framework}, it might be interesting to have a
look at the split statistics the split point estimate was derived from.
Those statistics can be extracted from the \texttt{splitstatistic} element
of a split and one can easily produce scatterplots against the selected
covariate. For all three inner nodes of \texttt{gtree}, we produce such a
plot in Figure \ref{glaucoma-split}. For the root node, the estimated split point
seems very natural, since the process of split statistics seems to have a
clear maximum indicating that the simple split point model is something
reasonable to assume here. This is less obvious for nodes $2$ and,
especially, $3$.

\begin{figure}[t]
\begin{center}
\begin{Schunk}
\begin{Sinput}
> cex <- 1.6
> inner <- nodes(gtree, 1:3)
> layout(matrix(1:length(inner), ncol = length(inner)))
> out <- sapply(inner, function(i) {
+     splitstat <- i$psplit$splitstatistic
+     x <- GlaucomaM[[i$psplit$variableName]][splitstat > 
+         0]
+     plot(x, splitstat[splitstat > 0], main = paste("Node", 
+         i$nodeID), xlab = i$psplit$variableName, 
+         ylab = "Statistic", ylim = c(0, 10), cex.axis = cex, 
+         cex.lab = cex, cex.main = cex)
+     abline(v = i$psplit$splitpoint, lty = 3)
+ })
\end{Sinput}
\end{Schunk}
\includegraphics{party-glaucoma-split-plot}
\caption{Split point estimation in each inner node. The process of 
         the standardized two-sample test statistics for each possible 
         split point in the selected input variable is show.
         The estimated split point is given as vertical dotted line.
         \label{glaucoma-split}}
\end{center}
\end{figure}

The class predictions of the tree for the learning sample (and for new
observations as well) can be computed using the \texttt{Predict} function. A
comparison with the true class memberships is done by
\begin{Schunk}
\begin{Sinput}
> table(Predict(gtree), GlaucomaM$Class)
\end{Sinput}
\begin{Soutput}
           glaucoma normal
  glaucoma       98     98
  normal          0      0
\end{Soutput}
\end{Schunk}
When we are interested in conditional class probabilities, the
\texttt{treeresponse} method must be used. A graphical representation is
shown in Figure \ref{glaucoma-probplot}.

\setkeys{Gin}{width=.6\textwidth}
\begin{figure}[ht!]
\begin{center}
\begin{Schunk}
\begin{Sinput}
> prob <- sapply(treeresponse(gtree), function(x) x[1]) + 
+     runif(nrow(GlaucomaM), min = -0.01, max = 0.01)
> splitvar <- nodes(gtree, 1)[[1]]$psplit$variableName
> plot(GlaucomaM[[splitvar]], prob, pch = as.numeric(GlaucomaM$Class), 
+     ylab = "Conditional Class Prob.", xlab = splitvar)
> abline(v = nodes(gtree, 1)[[1]]$psplit$splitpoint, 
+     lty = 2)
> legend(0.15, 0.7, pch = 1:2, legend = levels(GlaucomaM$Class), 
+     bty = "n")
\end{Sinput}
\end{Schunk}
\includegraphics{party-glaucoma-classprob}
\caption{Estimated conditional class probabilities (slightly jittered) 
         for the Glaucoma data depending on the first split variable. 
         The vertical line denotes the first split point. \label{glaucoma-probplot}}
\end{center}
\end{figure}

\paragraph{Node Positive Breast Cancer.}

\begin{Schunk}
\begin{Sinput}
> data(GBSG2, package = "ipred")
> stree <- ctree(Surv(time, cens) ~ ., data = GBSG2)
\end{Sinput}
\end{Schunk}

\setkeys{Gin}{width=\textwidth}
\begin{figure}[t]
\begin{center}
\begin{Schunk}
\begin{Sinput}
> plot(stree)
\end{Sinput}
\end{Schunk}
\includegraphics{party-GBSG2-plot}
\caption{Tree-structured survival model for the GBSG2 data and the distribution of 
survival times in the terminal nodes. The median survival time is displayed in
each terminal node of the tree. \label{gbsg2}}
\end{center}
\end{figure}  


Recursive partitioning for censored responses has attracted a lot of interest
\citep[e.g.,][]{regression:1988,relative-r:1992}. 
Survival trees using $P$-value adjusted Logrank statistics
are used by \cite{statistics:2001a} 
for the evaluation of prognostic factors for the German Breast
Cancer Study Group (GBSG2) data,  a prospective controlled clinical trial on the
treatment of node positive breast cancer patients. Here, we use
Logrank scores as well. Complete data of seven prognostic factors of $686$
women are used for prognostic modeling, the
dataset is available within the 
\texttt{ipred} package.
The number of positive lymph nodes
(\texttt{pnodes}) and the progesterone receptor (\texttt{progrec}) have
been identified as prognostic factors in the survival tree analysis
by \cite{statistics:2001a}.  Here, the binary variable coding whether a
hormonal therapy was applied or not (\texttt{horTh}) additionally is
part of the model depicted in Figure~\ref{gbsg2}.

The estimated median survival time for new patients is less informative
compared to the whole Kaplan-Meier curve estimated from the patients in the
learning sample for each terminal node. We can compute those `predictions'
by means of the \texttt{treeresponse} method
\begin{Schunk}
\begin{Sinput}
> treeresponse(stree, newdata = GBSG2[1:2, ])
\end{Sinput}
\begin{Soutput}
[[1]]
Call: survival:::survfit(formula = resp, weights = w[[i]], subset = w[[i]] > 
    0)

      n  events  median 0.95LCL 0.95UCL 
    248      88    2093    1814     Inf 

[[2]]
Call: survival:::survfit(formula = resp, weights = w[[i]], subset = w[[i]] > 
    0)

      n  events  median 0.95LCL 0.95UCL 
    166      77    1701    1174    2018 
\end{Soutput}
\end{Schunk}

\paragraph{Mammography Experience.}

\begin{Schunk}
\begin{Sinput}
> data(mammoexp)
> mtree <- ctree(ME ~ ., data = mammoexp)
\end{Sinput}
\end{Schunk}

\begin{figure}[t]
\begin{center}
\begin{Schunk}
\begin{Sinput}
> plot(mtree)
\end{Sinput}
\end{Schunk}
\includegraphics{party-mammo-plot}
\caption{Ordinal regression for the mammography experience data with the
fractions of (never, within a year, over one year) given in the nodes.
No admissible split was found for node 4 because only $5$ of $91$ women reported 
a family history of breast cancer and the sample size restrictions would 
require more than $5$ observations in each daughter node. \label{mammoexp}}
\end{center}
\end{figure}


Ordinal response variables are common in investigations where the response
is a subjective human interpretation. 
We use an example given by \cite{HosmerLemeshow2000}, p. 264, 
studying the relationship between the mammography experience (never,
within a year, over one year) and opinions about mammography expressed in
questionnaires answered by $n = 412$ women. The resulting partition based on
scores $\xi = (1,2,3)$ is given in Figure~\ref{mammoexp}. 
Women who (strongly) agree with the question `You do not need a mammogram unless
you develop symptoms' seldomly have experienced a mammography. The variable
\texttt{benefit} is a score with low values indicating a strong agreement with the
benefits of the examination. For those women in (strong) disagreement with the first
question above, low values of \texttt{benefit} identify persons being more likely to have 
experienced such an examination at all. 


\bibliographystyle{plainnat}
\bibliography{partyrefs}

\end{document}



\paragraph{Hunting Spiders.}

Finally, we take a closer look at a challenging dataset on animal abundance
first reported by \cite{AartSmeenk1975} and 
re-analyzed by \cite{Death2002} using regression trees dealing with
multivariate responses. The
abundance of $12$ hunting spider species is regressed on six
environmental variables (\texttt{water}, \texttt{sand}, \texttt{moss},
\texttt{reft}, \texttt{twigs} and \texttt{herbs}) for $n = 28$
observations. Because of the small sample size we allow for a split if at
least $5$ observations are element of a node and approximate the
distribution of a $c_\text{max}$ test statistic by $9999$ Monte-Carlo
replications. The prognostic factors \texttt{twigs} and \texttt{water}
found by \cite{Death2002} are confirmed by the model 
shown in Figure~\ref{spider} which additionally identifies \texttt{reft}.
The data are available in package 
\texttt{mvpart} \citep{mvpart2004} at \url{http://CRAN.R-project.org}.

\begin{Schunk}
\begin{Sinput}
> data(spider, package = "mvpart")
> sptree <- ctree(arct.lute + pard.lugu + zora.spin + 
+     pard.nigr + pard.pull + aulo.albi + troc.terr + 
+     alop.cune + pard.mont + alop.acce + alop.fabr + 
+     arct.peri ~ herbs + reft + moss + sand + twigs + 
+     water, controls = ctree_control(teststat = "maxabs", 
+     testtype = "MonteCarlo", nresample = 9999, minsplit = 5, 
+     mincriterion = 0.9), data = spider)
> sptree@tree$criterion
> library("coin")
> data(spider, package = "mvpart")
> it <- independence_test(arct.lute + pard.lugu + zora.spin + 
+     pard.nigr + pard.pull + aulo.albi + troc.terr + 
+     alop.cune + pard.mont + alop.acce + alop.fabr + 
+     arct.peri ~ herbs + reft + moss + sand + twigs + 
+     water, data = spider, distribution = approximate(B = 19999))
> statistic(it, "standardized")
> pvalue(it)
\end{Sinput}
\end{Schunk}

\begin{figure}[t]
\begin{center}
<<<<<<< party.Rnw
\begin{Schunk}
\begin{Sinput}
> plot(sptree, terminal_panel = node_terminal)
\end{Sinput}
\end{Schunk}
\caption{Regression tree for hunting spider abundance. The species with
         maximal abundance is given for each terminal node. \label{spider}}
\end{center}
\end{figure}



